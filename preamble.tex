\usepackage{array, epsfig, endnotes}
\usepackage{pifont}
\usepackage{wasysym, booktabs, rotating, multirow, adjustbox}
\usepackage{hyperref, cleveref,listings}
\usepackage{threeparttable}

%\usepackage[hyphens]{url}
%\usepackage{times}
\usepackage{textcomp}
%%!\usepackage{mathptmx}
\usepackage{epstopdf}
\usepackage{verbatim}
\usepackage{subfigure}
\usepackage{stmaryrd}
\usepackage{multirow}
\usepackage{listings}
\lstdefinelanguage[]{OblivC}[]{C}{morekeywords={obliv,obig,frozen,bool},sensitive=true,mathescape=true, numbers=left}
\newcommand\oblinline{\lstinline[language=OblivC,basicstyle=\small\sffamily,columns=fullflexible]}
\lstnewenvironment{OblivC}[1][]{\lstset{language=OblivC,%                                                    
    basicstyle=\small\sffamily,showstringspaces=false,columns=fullflexible,#1}}{}
\lstnewenvironment{PlainC}[1][]{\lstset{language=C,%                                                         
    basicstyle=\small\sffamily,showstringspaces=false,columns=fullflexible,#1}}{}
%\newcommand\obliv{\oblinline|obliv|}
\newcommand\oblivif{\oblinline|obliv if|}
\newcommand\frozen{\oblinline|frozen|}
\newcommand{\obliv}[1]{{\ensuremath\left\langle#1\right\rangle}}
\newcommand{\oblivelse}{\obliv{\textbf{else}}}
\newcommand{\arrayElt}[2]{{\ensuremath{#1}_{#2}}}
\newcommand{\oarrayElt}[2]{\obliv{{#1}}_{#2}}

\newcommand{\sfize}[2]{\newcommand#1{\ensuremath{\mathsf{#2}}\xspace}}
\newcommand{\osfize}[2]{\newcommand#1{\ensuremath{\obliv{\mathsf{#2}}}\xspace}}

\usepackage{siunitx}
\usepackage{color}
%\usepackage[sort]{cite}
\usepackage{graphicx}
\usepackage{graphics}
\usepackage{epsfig}
\usepackage{amsmath, amsfonts}
\usepackage{comment}
\usepackage{enumitem}
\usepackage[linesnumbered,ruled]{algorithm2e}
%\usepackage{algorithm}
%\usepackage{algorithmic}
\usepackage{algpseudocode}
\usepackage{varwidth}
\usepackage{listings}
\usepackage{graphicx}
\usepackage{xspace}
\usepackage{multirow}
%\usepackage{makeidx}
\usepackage{threeparttable}
\usepackage{booktabs}
\usepackage{url}
\usepackage{color}
%\usepackage[usenames,dvipsnames]{xcolor}
\usepackage{hhline}
\usepackage{makecell}
\usepackage{cryptocode}

\newcommand\shortsection[1]{\vspace*{6pt}{\noindent\bf #1.}}

\newcommand{\dnote}[1]{\textcolor{blue}{\bf \emph{Dave: #1}}}
\newcommand{\hnote}[1]{\textcolor{red}{\bf \emph{Haina: #1}}}
\newcommand{\bnote}[1]{\textcolor{purple}{\bf \emph{Bargav: #1}}}

\newcommand{\TODO}[1]{\textcolor{blue}{TODO:#1}}
\newcommand\keyword[1]{{\sf\small{#1}}}

%\algnewcommand{\AND}{\textbf{and}}
%\algnewcommand{\NOT}{\textbf{not}}

\newcommand{\todo}[1]{\textcolor{orange}{\bf \emph{TODO: #1}}}
\newcommand{\code}[1]{{\small\lstinline!#1!}}
\newcommand{\toolName}{SSOScan\xspace}
\newcommand{\var}[1]{{\ttfamily#1}}% variable


%\lstdefinelanguage{JavaScript}{
  %keywords={typeof, new, true, false, catch, function, return, null, catch, switch, var, if, in, while, do, else, case, break, eval, window},
  %keywordstyle=\color{blue}\bfseries,
  %ndkeywords={class, export, boolean, throw, implements, import, this},
  %ndkeywordstyle=\color{darkgray}\bfseries,
  %identifierstyle=\color{black},
  %sensitive=false,
  %comment=[l]{//},
  %morecomment=[s]{/*}{*/},
  %commentstyle=\color{purple}\ttfamily,	
  %stringstyle=\color{red}\ttfamily,
  %morestring=[b]',
  %morestring=[b]"
%}

%\lstset{language=JavaScript,
    %basicstyle=\sffamily,
    %stringstyle=\sffamily,
%    size=\small,
    %keywordstyle=\bfseries,
    %showstringspaces=false,
    %columns=fullflexible,             
    %morekeywords={document, printf}
%}

\newcommand{\subcap}[1]{\vskip 0.5ex\centering{\parbox{7.5cm}{\footnotesize #1}\vskip 1ex}}
\newcommand{\subcapw}[1]{\centering{\parbox{16cm}{\footnotesize #1}\vskip 1ex}}
\renewcommand{\textrightarrow}{$\rightarrow$}

\renewcommand{\topfraction}{0.9}	% max fraction of floats at top
\renewcommand{\bottomfraction}{0.8}	% max fraction of floats at bottom
    %   Parameters for TEXT pages (not float pages):
\setcounter{topnumber}{8}
\setcounter{bottomnumber}{8}
\setcounter{totalnumber}{10}     % 2 may work better
\setcounter{dbltopnumber}{10}    % for 2-column pages
\renewcommand{\dbltopfraction}{0.9}	% fit big float above 2-col. text
\renewcommand{\textfraction}{0.2}	% allow minimal text w. figs
    %   Parameters for FLOAT pages (not text pages):
\renewcommand{\floatpagefraction}{0.5}	% require fuller float pages
	% N.B.: floatpagefraction MUST be less than topfraction !!
\renewcommand{\dblfloatpagefraction}{0.5}	% require fuller float pages

\newcommand{\domain}[1]{{\tt #1}}
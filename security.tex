\section{Security Analysis}\label{sec:security}

\dnote{this section seems rambling now - is there anything new here that shouldn't be clear from earlier sections and belongs better here?  I think the scenario discussion belongs in 3.3, and I expanded that some when I read it earlier - this seems redundant mostly now}

ECDSA and other signing algorithms guarantee the authentication, non-repudiation, and integrity of digital documents as long as only the owner knows the private key. We discuss the security benefits of storing shares of the private key separately and computing the signing algorithm using MPC. 

Our decentralized certificate authority uses garbled circuits for public key generation and signing across two parties, delivering guarantees about security and privacy. The protocol reveals nothing but the outputs, the public key and the signature pair ($r,s$), so that one party cannot learn about the inputs, the private key and random key shares, of the other party. We implement Yao's protocol to provide security guarantees against semi-honest adversaries who adhere to the protocol but are curious to gain information about the inputs of the other party. For security against malicious adversaries, we implement Dual Execution protocol, which executes Yao's protocol twice and leaks only one bit of information. Dual Execution protocol detects and terminates the computation if one of the parties deviates from the protocol. Expansion to more sophisticated protocols with more than two parties is discussed in Section~\ref{sec:discussion}.

%Simply decentralizing the signature generation protects an all-powerful private key from being stolen, but adding independent parties (i.e. another CA or the subject) ensures that a certificate signature is not illegitimately generated. In the scenario where two CAs collaboratively generate a certificate, both must independently investigate the subject and come to the agreement that the certificate should be issued. The final scenario where the subject is involved in the certificate signing process guarantees the most protection against abuse because only the true owner of the domain would hold the private key to generate the signature.

In Section~\ref{sec:dca_setup}, we described three scenarios that could benefit from our decentralized certificate authority design. In all the scenarios, the voluntary or involuntary compromise of one machine would only reveal one share of the private key, which is meaningless without the other key share. When the private key shares are stored correctly on different servers in separate locations, a colluding personnel would not be able to leak the master private key or conduct illegitimate signing. The second scenario provides more assurance to subjects that the certificate and its contents are thoroughly checked before being issued. Two independent CAs' research committees or algorithms must come to the same conclusion when verifying the subject's identity and other pertinent information. For this plan to work, however, the two CAs must agree to collaborate with each other. Browser vendors are in the best positions to push for a decentralized CA model because they control which certificates are accepted by browsers.
By involving the client into the certificate generation process, only the domain holder with the correct private key share is capable of requesting for a new certificate. 

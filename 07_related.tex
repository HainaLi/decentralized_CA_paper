\section{Related Work}\label{sec:related_works}
We discuss related ideas on distributing the certificate authority and applying multi-party computation to AES.

\shortsection{Distributing Certificate Authority}
Decentralizing or distributing a certificate authority across many nodes is an idea first introduced by Zhou and Haas~\cite{zhou1999securing} in 1999 when tackling the problem of providing security for ad hoc, mobile networks where hosts relied on each other to keep the network connected. This work proposed spreading the functionality of a single certificate authority to a set of nodes using secret sharing~\cite{shamir1979share} and threshold cryptography. Later, researchers such as Dong et al.~\cite{dong2007providing} built upon this idea and provided practical deployment solutions. Though we do not consider mobile ad hoc networks, we attempt to solve the same problem through secure multi-party computation~\cite{yao1986generate}.

\shortsection{AES with Multi-Party Computation}
Multi-party computation has been used in many works~\cite{damgaard2010secure, launchbury2012efficient, laur2013oblivious, araki2016high} for joint AES computation among multiple participants. Damgard and Keller~\cite{damgaard2010secure} implemented three-party MPC protocol using VIFF framework for AES computation with honest majority for semi-honest adversary and achieved a block encryption in 2 seconds. The authors, however, claim that their method can be strengthened by using passively secure protocol~\cite{ben1988completeness} against dishonest majority or actively secure protocol~\cite{damgaard2009asynchronous} against a third of corrupted parties. Launchbury et al.~\cite{launchbury2012efficient} proposed a more efficient method for three-party AES encryption in the presence of semi-honest adversaries, where they replaced garbled circuits with look-up tables in the MPC protocol, and achieved 300 block encryptions per second. Laur et al.~\cite{laur2013oblivious} implement multi-party AES with secret-shared key and plaintext using Sharemind framework~\cite{bogdanov2008sharemind}, and then apply join operation over secret-shared tables in a database. Recently, Araki et al.~\cite{araki2016high} gave the first practical implementation of AES encryption with three-party protocol that provides security against semi-honest adversaries, and also guarantees privacy against malicious adversaries with honest majority. This privacy guarantee is weaker than security against malicious adversaries, in a sense, it only ensures that the malicious adversary does not infer about input or output of the computation, but can corrupt the computation, thereby affecting the correctness of the computation. Nonetheless, their method achieves 1 million AES operations per second. However, this work interests us the most since it also implements distributed ticket-granting ticket generation of Kerberos protocol under the same privacy settings, which is similar to our application of distributed certificate authority. Thus, extending the distributed certificate authority to three-party setting by adopting the method of Araki et al. is of particular interest, and is a motivation for future research.

%\dnote{need a much more scholarly and complete related work section - should include work on using MPC in similar contexts like http://eprint.iacr.org/2016/768.pdf on Kerberos, lots of previous work on split-key AES, etc.}




